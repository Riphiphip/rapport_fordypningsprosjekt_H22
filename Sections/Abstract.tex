\section{Abstract}
The Regionalized Value State Dependence Graph (RVSDG) is a relatively new compiler Intermediate Representation introduced by Reissmann et al., and is intended for use in the optimization stage of compilers \cite{Reissmann2018_multi-core, Reissmann2020}. It has several properties that are desirable for this use-case, but so far the only compiler implementation that uses RVSDG is a research compiler created to demonstrate its viability \cite{reissmann_github_2022}. The purpose of this project is to implement RVSDG as a dialect of MLIR \cite{mlir}. MLIR is a subproject of LLVM~\cite{llvm_homepage} which provides a state-of-the-art extensible compiler IR complete with out-of-the-box compiler infrastructure \cite{mlir}. This should provide a good platform for testing new optimizations and transformations on RVSDG IR and for comparing RVSDG to other compiler IRs. Having RVSDG implemented as an MLIR dialect should also make it easier for new compiler projects to leverage RVSDG for their own benefit.

So far, the main goal has been to get familiar with MLIR development patterns. This includes both figuring out how MLIR works conceptually and how to use it in practice. A part of this process was to find a way to map concepts from RVSDG onto MLIR. So far, the gamma-node has been implemented and some preliminary work on the lambda-node has been done. A seemingly good mapping from RVSDG to MLIR has also been found, which should make future implementation more manageable.

This report starts by providing an intro to LLVM, MLIR, and RVSDG. It then follows up by explaining what was done, what the outcomes of that work are, and how it could be improved. Finally, we discuss what the way forward will be for this project.